\documentclass[a4paper.12pt] {article}
\usepackage{graphicx}


\begin{document}
\title{CS6000 Journal Assignment 3}
\author{Brian D. Wisniewski}
\maketitle

\section{Process Overview}

My approach to quickly skimming these articles was derived from a technique I remember from a previous class many years ago.  The instructor had termed it "predatory reading" and it consisted of reading the Title, Authors (and author background if available for context) the abstract, introduction, or first paragraph, and then the first sentence and last 2 sentences of each paragraph through the conclusion.  Diagrams, drawings, and illustrations should be scanned and any questions noted.  

\section{Critical / Creative Reads}

1. Critical / Creative Reading - The author revisits the concept of the Maginot Line as originally envisioned.  Defense is meant to create opportunity for offensive action or in the case of cybersecurity, proactive measures. \cite{r3}
\newline
2. Critical / Creative Reading - The author outlines the requirements of international law in defining the potential of collateral damage with regards to cyberspace operations.  The article points out that many of the concerns around collateral damage are overblown and when compared to conventional secondary and tertiary impacts, may actually hold more options for military leaders to avoid unnecessary casualties. \cite{r5}
\newline
3. Critical / Creative Reading - Cyber Threat Characterization is important as we consider threat modeling and vulnerability assessment.  Numerous frameworks offer a way to assess the cyber threat in a standardized way.  The importance of incorporating these approaches into systems development early in the lifecycle is noted.\cite{r7}
\newline
4. Critical / Creative Reading - The authors describe 3 approaches to leveraging "manueverable applications."  Dynamic resource provisioning can utilize an approach similar to that used in academic environments for Hadoop scheduling. The second approach cites Software Defined Networking as a key capability required for application optimization.  Finally, creation of a distributed application environment offers the potential of a shifting attack surface.\cite{r9}
\newline
5. Critical / Creative Reading - The author argues that the traditional checklist approach must evovle.  It requires a more proactive approach where authoritative frameworks are used as the foundation, but organizations must apply unconventional controls on a dynamic basis based upon the evolving threat. \cite{r14}
\newline

\section{References}
\nocite{*}
\bibliographystyle{ieeetr}
\bibliography{references}

\end{document}